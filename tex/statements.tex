\chapter{Statements}

    \section{Expressions}
        An expression statement consists of an expression followed by a semicolon. Expressions in expression statements will be evaluated, and its value calculated.

        \begin{lstlisting}
a int = 500;
s char = 'a';
2 + 4 - 3;          /* Not used, thrown away */
        \end{lstlisting}


    \section{Declarations}
        A declaration specifies a variable's name and type, in that order. Values may also be initialized in the declaration.

        \begin{lstlisting}
x int;
y char = '4';
        \end{lstlisting}

    \section{Control Flow}

        \subsection{\texttt{if, elif, else}}
            An \texttt{if} statement causes a block (encapsulated by \texttt{\{ and\}}) to be entered if the specified condition evaluates to true.

            \vspace{5mm}
            \noindent An \texttt{elif} allows an alternate condition to be specified.
            
            \vspace{5mm}
            \noindent An \texttt{else} is entered if the `if' and `elif's are not entered. 
           
           \vspace{5mm}
            \noindent A boolean expression encapsulated within parentheses is required for every \texttt{if} and \texttt{elif}. \texttt{Elif} and \texttt{else} belong to the first preceeding \texttt{if} statement. 


            \begin{lstlisting}
x int = 1;
if (x == 1)
{
    print("x==1!");
}

elif (x == 2)
{
    print("x==2!");
}

else 
{
    print("fail");
}
            \end{lstlisting}

        \subsection{Looping with \texttt{for}}
            Democritus eliminates the \texttt{while} structure, replacing it instead with a modified \texttt{for} loop. \texttt{For} can be used to iterate by providing an initialization, termination condition, and update:
            \begin{lstlisting}
for(i int = 0; i < 10; i++)
{
    /* Some code here */
}
            \end{lstlisting}

            It can also be used as a while loop providing only one condition:

            \begin{lstlisting}
for(x < 10)
{
    /* Some code here */
}
            \end{lstlisting}


