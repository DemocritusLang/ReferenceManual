\chapter{Expressions and Operators}
	\section{Assignment}
		Assignment is done with \texttt{=}. As mentioned above, variables and declared with the \texttt{varname vartype} syntax. Variables can be assigned to a single value or to the result of an expression. 
		
		\begin{lstlisting}
x float = 4.0;
y int = 5/2 + 1; /* y = 3 */
		\end{lstlisting}
		
	\section{Arithmetic Operations}
		Democritus supports all the arithmetic operations standard to most general-purpose languages like C and Java. Note that casting is not built into the language; this functionality will instead be implemented through the standard library.
		
		\subsection{Addition and Subtraction}
			Addition works with the \texttt{+} character, behaving as expected.
			\begin{lstlisting}
x int = 4;
y int = 2;
x = x+y; 	/* x = 6  */
y = y-x 	/* y = -4 */
			\end{lstlisting}
						
		\subsection{Multiplication}
			\begin{lstlisting}
x int = 4;
y int = 2;
x = x*y+y; /* x = 10 */
			\end{lstlisting}
			
			\noindent Multiplication follows the same rules as well. 
		\subsection{Division}
			Democritus will default to integer division, unless both types provided are floats. 
			\begin{lstlisting}
x int = 5;
y int = 2;
x = x/y; 	/* x = 2  */

a float = 4.0;
b float = 2.0;
a = b/a; 	/* a = 2.0 approximately */
			\end{lstlisting}
			
	\section{Boolean Expression}
		Democritus 