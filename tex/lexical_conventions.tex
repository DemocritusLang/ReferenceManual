\chapter{Lexical Conventions}
	In this section, we will cover the standard lexical conventions for Democritus. Similarly to languages such as C, Algol, or Pacal, Democritus is a free-format language. Thus, the parser will discard whitespace characters such as \texttt{` '}, \texttt{\textbackslash t}, and \texttt{\textbackslash n}.
	
	\section{Identifiers}
		Identifiers for Democritus will be defined in the same way as they are in most other languages; any sequence of letters and numbers without whitespaces and is not a keyword will be parsed as an identifier. Note that, as in other languages, identifiers cannot begin with a number. Somewhat different, however, is the order of variable declarations; in Democritus, declarations are made following the \textit{varname vartype} structure. 
		
	\begin{lstlisting}
2wrongID int; 	/* Not a valid identifier */
mySecond float;	/* Valid */
my_Second char; /* Valid */
	\end{lstlisting}
	
	\section{Keywords}
		The list of reserved keywords used in Democritus are as follows:
		\begin{lstlisting}[language={}]
if
else
elif
for
return
int
float
char
boolean
function
void
string
true
false
break
continue
atomic
		\end{lstlisting}
		\noindent These words have been reserved by the compiler and hold special meaning within the language. Though most are self-explanatory, we will delve into their usage later on. 
		
	\section{Punctuation}
		\subsection{;}
			Similarly to C, the semicolon `;' is required to terminate any statement in Democritus. 
		\subsection{\{ and \}}
			For the sake of keeping the language free-format, curly braces are used to delineate between separate and nested blocks. These braces are required even for single-statement conditional and iteration loops. 
		\subsection{( and )}
			To assert precedence, expressions may be encapsulated within parentheses to guarantee order of operations. 
		\subsection{Comments}
			For now, comments are initiated with \texttt{/*} and closed with \texttt{*/}. They cannot be nested.

			
			
			
