\chapter{Data types}
    \section{Primitive Types}
        \subsection{int}
            A standard 32-bit two's-complement signed integer. It can take any value in the inclusive range (-2147483648, 2147483647).
        \subsection{float}
            A 64-bit floating precision number, represnted in the IEEE 754 format.            
        \subsection{char}
            An 8-bit ASCII character.  
        \subsection{boolean}
            A 1-bit boolean may take a true or false value. 
        \subsection{pointer}
            An 64-bit pointer holds the value to a location in memory; they operate similarly to those found in C.
    \section{Complex Types}
        \subsection{Array}
            A fixed-size array, allocated on the stack (thus requiring the size to be defined at declaration) of other primitive types. An array object can be accessed by C array notation, such as \texttt{list1[0]}. 
 \iffalse       \subsection{struct}
            A struct is a simple user-defined data structure that holds various primitives, similar to the ones found in C. 
 \fi
