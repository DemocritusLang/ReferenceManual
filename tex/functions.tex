\chapter{Functions}
    \section{Overview}
        Functions can be defined in Democritus to return one or no data type.  Functions are evaluated via eager evaluation and the function implementation must directly follow the function header.
        
        \vspace{5mm}
        \noindent A function appears in the form:
        
        \begin{lstlisting}
function [function name]([type:formal_arg, ... ]):[return type]{
    [function implementation]
    return [variable of return type]
}
        \end{lstlisting}

        \noindent \textbf{Note}: all functions need \texttt{return} statements at the end (no falling off the end). A void \texttt{return} is simply a return with nothing following it.

        \vspace{5mm}
        \noindent Functions may be recursive and call themselves:

        \begin{lstlisting}
function recursive_func(i:int):void{
    if(i < 0){
        return;
    }else{   
        print “hi”;
        recursive_fun(i-1);
    }
}
        \end{lstlisting}


        \noindent Functions may be called within other functions:
        \begin{lstlisting}

function main():void{
    recursive(3);
    return;
}
        \end{lstlisting}


    \section{Built-in Functions}
        A handful of functions are natively built into Democritus for user flexibility and ease of usage. There are:
        \begin{itemize}
            \item print(s:string): takes in a string (standard library functions will convert from other data types to strings)
            \item thread(f:function, [arg1:type, arg2:type, ...]): takes in function and function args
        \end{itemize}

